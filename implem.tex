
A \texttt{TCPLS} reference implementation is under active development. The
current implementation is forked from a fast and full
TLS 1.3 implementation written in \texttt{C}. It currently adds about 5k lines
of code (including unit tests and inline code comments) compared to the upstream branch.

The current implementation offers: 
\begin{inparaenum}
  \item An experimental API that wraps TLS and TCP
  \item Our design of TCP's extensibility mechanism throught TCP option sent
within TLS records and processed by TCPLS. We currently support TCP User
Timeout and the injection of eBPF bytecode for TCP's congestion control
algorithm. Supporting another TCP option is only a matter of extending the sender's
API and processing the option
on the receiver side. TCPLS's internal machinery is already implemented for any
type of TCP option to send during the handshake or when the handshake completed.
  \item The support of parallel streams and multiplexing over TCP connections.
    Each stream has its own cryptographic context.
  \item Connection migration and multipathing.
\end{inparaenum}

We expect to investigate several research questions with TCPLS. First, we are
interested in analyzing how far we can go into supporting TCP's extensibility
through our mechanism. Several of new TCP features may require to inject eBPF
bytecode to the kernel. It is still unclear how much of TCP could be extended,
and if \texttt{TCPLS} may play a role in incentivizing the linux kernel's maintainer
to support more eBPF in TCP's implementation, since we now offer a technique to
propagate eBPF bytecode within authenticated and confidential session with a
trusted server.

Second, we expect to analyze several features of TCPLS, such as our
connection migration, our multipath implementation and our failover mechanism.
Answering the question about how much TCPLS can or and how fast it performs in
case of a migration or a network outage.

Finally, we would expect to critisize QUIC in light of our \texttt{TCPLS}
design. We will compare both QUIC and TCPLS, discuss the differences, and analyze
and answer questions regarding the requirements in transport protocol for today's
Internet applications.



