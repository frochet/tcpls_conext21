% !TEX root = ./paper.tex
\tcp and \tls were designed as independent protocols, but they are very often
used together. In this paper, we have shown that this is possible to design and
implement the \tcpls protocol that is fast, flexible, and secure by closely
coupling \tcp and \tls.

\tcpls inherits the security features of \tls 1.3 and all the reliability and
congestion control techniques that have been added to \tcp during the last
decades. More specifically, \tcpls extends the \tls 1.3 handshake and the record
layer to create a secure control channel between the two communicating hosts.
The messages exchanged over this channel are placed inside \tls records that are
encrypted and authenticated and also hidden from middleboxes. \tcpls leverages
these control messages to support fast failovers, but also offer smooth
migration of the \tcpls session from one path to another or provide bandwidth
aggregation under full control of the application through an API. \tcpls can
also use the secure channel to extend \tcp with new options and even use the
\tcpls session to exchange a different congestion control scheme that is then
used for this session.

Thanks to \tcpls's design, our \tcpls implementation provides a
flexible API that allows the applications to perform zero-copy data transfers,
and easily manipulate the different features presented in this research work.
Our performance evaluation shows that this prototype is more than twice as fast
as currently available \quic implementations while already supporting additional
features such as bandwidth aggregation and stream steering.

\tcpls is both simple and powerful. Simple because it can be implemented inside
existing \tls libraries without any kernel change in contrast to \tcp. \tcpls is
wire-compatible with the existing \tcp middleboxes and can thus be used in all
environments where \tls is used over \tcp. Given the performance benefits of
\tcpls, the flexibility it offers and the features that it already provides, we
believe it can be a strong contender to \quic for modern services, including
HTTP.
