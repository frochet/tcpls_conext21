% !TEX root = ./paper.tex 

% testbed description
 We deploy a testbed composed of three machines with Intel Xeon CPU E5-2630
2.40GHz, 16 Threads, at least 16GB RAM, running Debian 9.0 with 4.9 and
4.12 kernels. Two of these machines play the role of Client and Server,
while one is the Network Simulator (NS). Each machine is equipped with an
Intel XL710 2x40GB NIC, directly connected as shown in Fig.~\fig{testbed}.
Traffic exchanged by client and server has to go through the NS first.

% More details on the NS
The NS relies on \dfn{Vector Packet Processing} (VPP)~\cite{vpp}, a
kernel-bypass packet processing framework, and on \mmb and \texttt{nsim}
plugins for simulating realistic network conditions and middlebox
interferences. \mmb is a middlebox plugin for VPP that allows to build various
stateless and stateful classification and rewriting middlebox
policies~\cite{edeline2019mmb}. It is used to recreate existing middlebox
traffic impairments without introducing additional processing overhead.
\texttt{nsim} is a network delay simulator that adds delay and shapes traffic
by processing packet vectors. The NS device is configured to maximize
its performances, to make sure that it is not the measurements bottleneck.
