% !TEX root = ./paper.tex
The Transmission Control Protocol (\tcp) \cite{rfc793} is one of the most
critical protocols in today's Internet. It has been designed following a
layer approach and now serves a wide range of
applications. During the last four decades, \tcp evolved under
the pressure of competing protocols. During the 1980s, software-based \tcp
implementations were considered too slow. Researchers proposed new transport
protocols such as \xtp~\cite{sanders1990xpress} which could be implemented in
hardware. Meanwhile, \tcp implementations got a considerable speed
boost~\cite{clark1989analysis}, and \xtp disappeared. The \tcp speed boost and
usage triggered the development of various important \tcp extensions, including
timestamps and large windows~\cite{rfc1323} or Selective
Acknowledgments~\cite{rfc2018}.

In the mid-nineties, the Secure Socket Layer (SSL) protocol was proposed as an
additional layer to \tcp to secure emerging e-commerce 
websites~\cite{draft-hickman-netscape-ssl}. SSL evolved in different versions 
of the Transport Layer Security (\tls) protocol, the most recent one being 
version 1.3~\cite{rfc8446}. %Many details of the \tls protocol have changed 
%since the first version of SSL~\cite{kotzias2018coming}. 
Nowadays, \tls is almost ubiquitous on web servers~\cite{holz2019era} and many 
non-web applications use it~\cite{anderson2019tls}.

During the late nineties, early 2000s, transport protocol researchers explored
alternatives to \tcp. The IETF standardized two new transport protocols:
\dccp~\cite{kohler2006designing} and \sctp~\cite{rfc4960}. We rarely use \dccp
today. Despite \sctp benefits (support for multihoming, better design, and
extensibility), only niche applications use it. %This limited deployment is
%probably due to two different factors. First, \sctp forced the applications to 
%%%<--(mp): c'est aussi ce qu'a fait quic et tcpls
%use a new API. 
This limited deployment is mainly due to the various middleboxes (NAT,
firewalls, etc.) deployed in 
the Internet that often block packets that do not carry \tcp or
\udp~\cite{honda2011still}.  \sctp initially supported multihoming by switching
from one path to another. It was later extended to use different
paths continuously~\cite{iyengar2006concurrent}. Multipath
\tcp~\cite{rfc6824,raiciu2012hard} brought similar capabilities to \tcp, and
included a coupled congestion control scheme~\cite{wischik2011design}, later
brought to \sctp as well. This particular succession of events shows how
different designs compete and advance each other.

Extending \tcp today is not feasible anymore as middleboxes severely interfere 
with changes to the \tcp header and 
options~\cite{medina2004measuring,honda2011still, edeline2019bottom}.
To overcome this problem, Google started \quic as an experimental 
protocol~\cite{roskind2013quic,langley2017quic} combining functions usually 
found in \tcp, \tls and HTTP/2. During the last years, it
evolved into a complete transport protocol~\cite{rfc9000}.
%whose standardization is being finalized within the IETF
\quic leverages encryption to prevent middlebox interference and propose to 
revisit the layered model of the Internet to improve the transport services.
%Indeed, the integration of \tls 1.3 allows 1-RTT secure 
%handshakes and extensive packet encryption, providing more security and 
%preventing middlebox interference.
%A key characteristic of \quic is that it encrypts almost all the packets, including most of their
%headers.
As \quic runs atop \udp, it can be implemented and deployed as a user-space library.
%Although \quic is essentially a new transport protocol, it does not run
%directly above IP in contrast with \sctp or \tcp. \quic runs above \udp, enabling
%user-space implementation as a library and overcoming middleboxes that filter IP.
%With this choice, \quic can be implemented as a user-space library, and \quic packets
%can pass through middleboxes.

Does the standardization of \quic marks the end of the \tcp era, moving
all applications and transport research to \quic?  We do not think
so. Today, QUIC is mainly used for HTTP/3~\cite{http3} and TCP remains a 
fallback because of its greater support in networks. TCP also still serves many 
applications~\cite{covid19,fiveyears}.
%In the future, TCP will remain the fallback for QUIC because of its greater 
%supportin networks.
%
%History tells us that \tcp has evolved with competing transport protocols.
In the light of those recent advances, we revisit how transport services can be
provided with \tcp and \tls today. The design of QUIC integrates services
that were found in the security and application layers, e.g., encryption and multiplexing.
\tcp and \tls have been both designed in strict layers separating the two.
%\quic is today's competitor and but there is still plenty of room to improve \tcp.
This paper revisits this separation through the lens of the following research questions:

\begin{itemize}
	\item[{\small{\textit{RQ1}}} -] How can TCP and TLS be 
	combined to improve extensibility and middlebox resilience ?
  %\todo{Discuss middleboxes and TCP extensibility so that the 
	%question can become: How can TCP and TLS be combined to alleviate middlebox 
	%interference or smth.}
%  \item[{\small{\textit{RQ2}}} -] How can we alleviate middlebox
%    interferences?
	\item[{\small{\textit{RQ2}}} -] What are the new transport services this 
	combination can offer?
	%\item[{\small{\textit{RQ3}}} -] How do these services compete with other
	%protocols ?
\end{itemize}

To answer these questions, we design and implement an approach that combines
\tcp and \tls 1.3 into a fast, flexible, and secure transport protocol called \textbf{\tcpls}
%
%In this paper, we take a step back. As \quic closely couples the reliability and
%security mechanisms, we reconsider the separation between \tcp and \tls.  In
%this paper, we combine both \tcp and \tls 1.3 in a single fast, flexible, and
%secure protocol called \textbf{\tcpls}
\footnote{A preliminary version of this work has been presentend in a workshop
  paper~\cite{rochet2020tcpls}.}.  At the heart of our approach, we illustrate
how \tls records (i.e., messages exchanged through \tls) can be leveraged to 
build new transport services. In \tcp/\tls, \tls \texttt{AppData} records are 
solely used to securely convey the \tcp bytestream. In \tcpls, we extend their 
use to convey both application and control data including encrypted \tcp 
options.
%\todo{more detail needed?}

We demonstrate that this combination allows secure extensibility that can also
be used with techniques such as TCP Fast Open \cite{rfc7413} to lower the
handshake latency. We leverage this new extensibility to implement modern transport
services such as multiplexing, connection migration, and stream steering capabilities
without risking middlebox interference. Our \tcpls prototype is implemented as a
user-space library exposing a powerful API to applications % around a Stream
%Steering mechanic
while leveraging the
high-performance Linux kernel \tcp stack. Our lab measurements indicate that
\tcpls can be implemented at a low cost while providing more bulk throughput and
features than the \quic implementations we tested.
%(mp): Je n'aime pas trop cette notion de flexibilité que l'on ne définit jamais vraiment

%We have
%designed \tcpls with three goals in mind. First, \tcpls exposes modern transport
%features, such as multipath capabilities, to the application. Second, \tcpls
%solves \tcp's extensibility issues in \tcp by relying on the \tls handshake for
%\tcp options negotiation and by including \tcp options inside \tls records.
%Finally, we draw a path to make \tcpls an excellent challenger to \quic.  With that in mind, we have implemented \tcpls as a user-space library that exposes a powerful API to applications but still relies on the high-performance in-kernel \tcp stack. Our implementation uses \tls' flexible record layer to create a secure control channel that the \tcpls session endpoints can use to exchange
%control information. This channel enables different use cases such as connection
%migration or seamless failovers without risking middlebox interference.

%Our lab
%measurements indicate that our \tcpls prototype outperforms existing \quic
%implementations in terms of raw performance while allowing more flexibility
%for recovery and connection migration. Our \tcpls has been thought
%with open access in mind and will be released upon this paper acceptance.

The rest of this paper is organized as follows: Section~\ref{sec:background}
provides the required technical background; Section~\ref{sec:background-design}
discusses how we designed \tcpls, while Section~\ref{sec:prototype} focuses on 
how
we implemented \tcpls. Section~\ref{sec:evaluation} evaluates the performance 
and
behavior of \tcpls. Section~\ref{sec:related} discusses the related work.  
finally, Section~\ref{sec:conclusion} concludes this paper by
summarizing its main achievements and discussing further directions.
%This work does not raise any ethical issues.
