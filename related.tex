% !TEX root = ./paper.tex

By combining \tcp and \tls, \tcpls builds upon two of the most important
Internet protocols. Given their importance within 
the research community and the IETF, we restrict our discussion to close
related works. Readers may refer to survey papers for additional 
context 
information~\cite{polese2019survey,li2016multipath,papastergiou2016ossifying}.

Given its security features, \tcpls must be compared with 
\tcpcrypt~\cite{bittau2010case,rfc8548} which predates \tls 1.3. It uses \tcp 
options to support opportunistic encryption but is not secure against an active 
network attacker. \tcpls extends \tls while retaining its security 
properties, and supporting new features. It is also compatible with \tcp 
middleboxes, as it leaves the \tcp wire format untouched.

\tcpls also needs to be compared with \mptcp \cite{raiciu2012hard,rfc6824}.
\mptcp supports several coupled congestion control schemes
\cite{peng2014multipath,wischik2011design,khalili2013mptcp} that preserve
fairness when different paths share the same bottleneck. A similar solution 
could be applied to \tcpls by leveraging eBPF to access and modify the 
congestion controller state. We leave this engineering effort as future work.
The initial idea of coupling \mptcp and \tls was proposed in an
Internet draft~\cite{draft-paasch-mptcp-ssl-00} that was not adopted by the
IETF. MPTCPSec~\cite{jadin2017securing} adds security capabilities to \mptcp but
comes with a large performance penalty.

Another important related work is \quic. Its design has 
evolved over the years, and several companies adopted 
it~\cite{langley2017quic,Joras_mvfst,marx2020same}. \quic
version 1~\cite{rfc9000} also supports connection
migration, but to our knowledge no implementation offers yet this feature to 
applications outside from limited interoperability scripts.
%we could not test it in our lab since none of 
%the available open-source implementations have implemented it yet. 
Multipath extensions
\cite{viernickel2018multipath,de2017multipath,draft-liu-multipath-quic-02}
to \quic have been discussed but not yet adopted within the IETF. Finally,
PQUIC~\cite{de2019pluginizing} proposed to convey eBPF code over \quic
connections to deploy new protocol features. This goes beyond the exchange of
congestion controller in \tcpls.

Finally, several solutions have been proposed to provide multipath capabilities
in the application layer. Examples include MP-H2~\cite{nikravesh2019mp} that
extends HTTP/2, MP-DASH~\cite{han2016mp} for video streaming or mHTTP~\cite{kim2014multi} that extends HTTP.

\tcpls also share similarities with \tls FOP~\cite{sy2020enhanced}, which 
solves the privacy issue with TFO by using TLS. However, \tcpls is more generic,
and demonstrate how we leverage its extensibility work for more advanced 
concerns and transport services.

\todo{Structured stream}

%% Several researchers have proposed techniques to extend various transport
%% protocols. The IETF provides generic guidelines on the design of
%% protocol extensions \cite{rfc6709}. Several researchers have proposed
%% solutions to simplify the implementation of extensions to transport
%% protocols. The closest example include STP \cite{patel2003upgrading}, CTP
%% \cite{bridges2007configurable} and

%% Patel et al. \cite{patel2003upgrading} propose to use a type-safe version of C to extend a TCP
%% implementation by using bytecode. The implementation of TCPLS completely differs since it uses C to generate eBPF code. Furthemore, by leveraging the
%% flexibility of TLS, TCPLS can securrely exchange various options over a
%% connection. Bridges et al.
%% In CTP, Bridges et al. propose a new protocol that is composed of different microprotocols which can be combined together. In PQUIC, De Coninck et al. \cite{de2019pluginizing} include an eBPF virtual machine inside a QUIC implementation to extend it by using protocol plugins. Our approach is similar from an implementation viewpoint. By combining the TLS and TCP layers, we bring more extensibility to TCP.

%% %maybe
%% % ictcp \cite{wong2001configurable}

%% %configurable and extensible transport \cite{wong2001configurable}

%% % other protocols maybe

%% The IETF has developed various transport protocols, including
%% DCCP \cite{kohler2006designing}, SCTP \cite{rfc4960} and QUIC \cite{draft-ietf-quic-transport}. DCCP brings more flexible congestion control schemes

%% , lots of research but nothing related \cite{nowlan2012fitting}




%% % maybe
%% % Structured streams \cite{ford2007structured}


%% %TCP papers
%% %generic on extensibility

%% %tcp unreliable

%% A wide range of TCP extensions have been proposed \cite{rfc7414}. Some tune
%% protocol implementations with new strategies to retransmit lost data,
%% compute retransmission timers or manage congestion. These do not require
%% the definition of new TCP options that are negotiated during the handshake.
%% Some TCP extensions require the definition of new TCP options. These
%% include the timestamp and large windows extension \cite{rfc7313},
%% the support for selective acknowledgements \cite{rfc2018}, TCP Fast Open
%% \cite{rfc7413} or Multipath TCP \cite{rfc6824}. These extensions are negotiated
%% by exchanging TCP options during the connection handshake. These TCP options
%% allow to extend TCP, but they suffer from several limitations. First,
%% there is limited space in the TCP header to carry them. This limits the
%% number of extensions that can be used for a given TCP connection. The IETF
%% discusses solutions to extend the TCP header space, but the solution is neither
%% finalized nor implemented \cite{draft-ietf-tcpm-tcp-edo-10}. Second, middleboxes
%% interfere with the utilization of new options \cite{honda2011still}. This
%% interference severely limits the extensibility of TCP. TCPLS does not suffer
%% from these problems since the options that it carries are encrypted and can
%% span an entire TLS record (16KBytes).


%% % \cite{nowlan2012fitting}
%% %TODO
%% % idée de combiner des infos de TLS dans MPTCP
%% %\cite{draft-paasch-mptcp-ssl-00}
%% %\cite{jadin2017securing} % secure mptcp
