% !TEX root = ./paper.tex
Several researchers have proposed techniques to extend various transport
protocols. The IETF provides generic guidelines on the design of
protocol extensions \cite{rfc6709}. Several researchers have proposed
solutions to simplify the implementation of extensions to transport
protocols. The closest example include STP \cite{patel2003upgrading}, CTP
\cite{bridges2007configurable} and PQUIC \cite{de2019pluginizing}.

Patel et al. \cite{patel2003upgrading} propose to use a type-safe version of C to extend a TCP
implementation by using bytecode. The implementation of TCPLS completely differs since it uses C to generate eBPF code. Furthemore, by leveraging the
flexibility of TLS, TCPLS can securrely exchange various options over a
connection. Bridges et al.
In CTP, Bridges et al. propose a new protocol that is composed of different microprotocols which can be combined together. In PQUIC, De Coninck et al. \cite{de2019pluginizing} include an eBPF virtual machine inside a QUIC implementation to extend it by using protocol plugins. Our approach is similar from an implementation viewpoint. By combining the TLS and TCP layers, we bring more extensibility to TCP.

%maybe
% ictcp \cite{wong2001configurable}

%configurable and extensible transport \cite{wong2001configurable}

% other protocols maybe

The IETF has developed various transport protocols, including
DCCP \cite{kohler2006designing}, SCTP \cite{rfc4960} and QUIC \cite{draft-ietf-quic-transport}. DCCP brings more flexible congestion control schemes

, lots of research but nothing related \cite{nowlan2012fitting}




% maybe
% Structured streams \cite{ford2007structured}


%TCP papers
%generic on extensibility

%tcp unreliable

A wide range of TCP extensions have been proposed \cite{rfc7414}. Some tune
protocol implementations with new strategies to retransmit lost data,
compute retransmission timers or manage congestion. These do not require
the definition of new TCP options that are negotiated during the handshake.
Some TCP extensions require the definition of new TCP options. These
include the timestamp and large windows extension \cite{rfc7313},
the support for selective acknowledgements \cite{rfc2018}, TCP Fast Open
\cite{rfc7413} or Multipath TCP \cite{rfc6824}. These extensions are negotiated
by exchanging TCP options during the connection handshake. These TCP options
allow to extend TCP, but they suffer from several limitations. First,
there is limited space in the TCP header to carry them. This limits the
number of extensions that can be used for a given TCP connection. The IETF
discusses solutions to extend the TCP header space, but the solution is neither
finalized nor implemented \cite{draft-ietf-tcpm-tcp-edo-10}. Second, middleboxes
interfere with the utilization of new options \cite{honda2011still}. This
interference severely limits the extensibility of TCP. TCPLS does not suffer
from these problems since the options that it carries are encrypted and can
span an entire TLS record (16KBytes).


% \cite{nowlan2012fitting}
%TODO
% idée de combiner des infos de TLS dans MPTCP
\cite{draft-paasch-mptcp-ssl-00}
\cite{jadin2017securing} % secure mptcp