% !TEX root = ./paper.tex
\label{sec:content}
In this section, we describe our \tcpls implementation.  In a nutshell, our implementation offers the following benefits:
($i$) The support of parallel streams and multiplexing over \tcp connections
  with different cryptographic context.
($ii$) An experimental API that wraps \tls and \tcp and enables applications to
    handle multihoming, multipathing, and various transport layer mechanisms.
($iii$) An improved \tcp extensibility mechanism that sends \tcp options
   through the secure \tcpls channel (as described in
   Sec.~\ref{sec:background-design}). We currently support Linux's \tcp User Timeout
   option. Supporting another \tcp option is only a matter of extending the
   sender's API and processing the option on the receiver side. \tcpls's
   internal machinery can already send any \tcp option during (server-side) or
   after (client-side) the handshake.
($iv$) Different multipath modes for the \tcpls streams: bandwidth aggregation or
independant paths.
($v$) The ability for the server to send eBPF bytecode over the secure
  channel to upgrade the client's \tcp congestion control scheme or
  tune other \tcp mechanisms~\cite{brakmo2017tcp, tran2019beyond}.
  Variable-length options (e.g., eBPF bytecode) can be sent within streams to
  take advantage of bandwidth aggregation or Failover if those features are
  used.
($vi$) Different Connection Migration modes: Failover and Application-level
Connection Migration.


%\fr{I don't want to give the impression that everything works like a charm -- I
%would need 1 more year to have something production ready}
%We stress that all of these features have been tested but this prototype is a
%research-level implementation. The implementation can
%showcase those features but is however not yet ready for production, as many
%bugs likely remain despite our unit tests and integration tests. Portability was
%neither part of our goals, but might be necessary for a production ready \tcpls
%library.  

\subsection{The \tcpls API}

The API that applications use to interact with a protocol plays an important
role in enabling them to leverage all the protocol features. The most popular
API to interact with the transport layer remains the BSD socket API. Researchers
and the IETF have explored new ways to expose a transport
API~\cite{draft-ietf-taps-arch,hruby2014sockets,rfc6458,hesmans2016enhanced,schmidt2013socket}.
We show in appendix~\ref{appendix:api} a simple example of our API workflow,
which develop common good practices proposed by the outlined research.

%In this spirit, application-level developers would only be required to
%configure a \tcpls context and register function callbacks.
%We design \tcpls such
%that the application-level developers can ignore any notion of Network IPC as
%defined by, for example, the POSIX API, the Berkeley socket API or Winsock,
%facilitating application-level development by offering a more concrete
%session-level interface based on asynchronous network events.
%The overall idea is to offer to application developers the opportunity to tune
%the transport protocol for a better usage of the network from their own
%application protocol, which might depend on its distinguishing
%features.
%\todo{we need to explain the mpjoin}


%Note, those features are not stable yet, and many bugs remain to be fixed.

\subsection{Multipathing}

\subsubsection{Data Aggregation}
The application can connect and join multiple \tcp connection to
the same \tcpls session. As soon as one of the peers attaches streams to
different \tcp connections of the session, and enables the aggregation mode,
\tcpls adds a sequence number encrypted in the \tls record payload. This
sequence number is used to reorder the received records after decryption. Our
current state of implementation supports one global ordering, but
we envision for the \tcpls protocol to have streams potentially detached from
the global orderings. For example, a HTTP application may want to use an
aggregation mode for 2 streams over 2 \tcp connections downloading a video
content for the video playback engine, but the other streams used by the HTTP
client do not necessarly need to be part of the multipath bandwidth aggregation.
Such a feature may be implemented through a negotiation of the aggregation
mode.

The implementation currently supports a round-robin scheduler. We expect in the
future that the receiver's scheduler might set through the API, or even sent
from the peer to enable an efficient scheduling against particular sending patterns. The more
the \tcpls receives the records in order, the more \tcpls can deliver them in a zero-copy
fashion to the application. When some record number is not the next expected
record, there is a copy of the record content made within a reordering buffer.

\subsubsection{Failover}\label{failover}
Failover is a binary mode (on/off) fully internal to \tcpls. Once activated,
\tcpls exchanges acknowledgments for records received in each stream. The
acknowldgment is stream-based and configurable. The default acknowledgment
policy sends an ack every 16 received records, or when a stream has process more
than $249,600$ bytes since the last acknowledgment.  When a \tcp connection
sustaining \tcpls streams suffers from a network outage (e.g., a $RST$ is
received or the connection became idle for too long), we move the stream to a
new \tcp connection, retransmit the unacked records and resume the transfer. Our
prototype handles failover over v4 or v6 \tcp connection, and by default,
chooses a different source and destination address than the failed \tcp
connection if some are available. Failover might be negotiated by both party, or
enabled by default by both party. Depending on the application type, Failover
might be enabled or not by default and a negotiatiation through \tcpls's control
channel should happen to change its value (on/off).

%\subsubsection{Application-level Connection Migration}
%\label{sec:connmigr}


%When an application feels right to migrate its connection, it
%can follow those simple steps: activating the multipath aggregation
%mode, then making a \tcpls join handshake on the new path. Then, opening a new stream
%and attaching it to the new \tcp connection, and closing the initial
%stream would make the data transfert enter in a temporary two-paths aggregated mode in
%which the other peer's first path flushes its data if any, and then gracefully close the \tcp
%connection achieving a smooth migration. In practice, such a migration would
%achieve better goodput than a QUIC single-path migration design in which the
%data path is temporally broken and then recovered. In our design, the
%application can make such a migration in 5 API calls.

%\subsection{TCP Options and Kernel extensibility}

\subsection{QUIC-like roundtrips}

One argument for QUIC's usage on the web was a quickee.
Compared to TLS/TCP, QUIC has only one handshake and can then proceed with the
application data. $TLS/TCP$ has two: First, the \tcp three-way handshake, and
then the \tls handshake. \tcpls can use \tcp's TFO~\cite{radhakrishnan2011tcp}
and send the \texttt{ClientHello} message within the \tcp SYN's payload,
achieving the same roundtrips than QUIC. However, TFO suffers from privacy
issues~\cite{sy2020enhanced}, thus we did not enable it by default. In the
future, we may expect to revise our choice if a solution similar to \tcp
FOP~\cite{sy2020enhanced} gets implemented in the Linux kernel.


