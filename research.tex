% !TEX root = ./paper.tex
\label{sec:research}
By closely integrating \tcp and \tls, \tcpls opens new research questions
in the transport layer and above. We highlight some of these in this section.

\subsection{A More Secure Multipath TCP}

Multipath TCP \cite{rfc6824, rfc8684} is a recent TCP extension that allows a connection
to send data over different paths. It defines several \tcp options, including
\texttt{ADD\_ADDR} to advertise addresses and \texttt{RM\_ADDR} to remove addresses. Thanks to the \texttt{ADD\_ADDR} option, a dual-stack server can advertise
its IPv6 address over an IPv4 connection initiated by the client. The client can
then use this address to create an IPv6 subflow that is part of the same
connection.

One of the major deployments of Multipath TCP is on Apple's iPhones
\cite{bonaventure2016multipath}. This implementation has decided not to
support the \texttt{ADD\_ADDR} option for security reasons. Since the
Multipath TCP options are sent in clear, an attacker or
a malicious middlebox could try to hijack connections.
%The latest version
%of Multipath \tcp \cite{rfc8684} addresses this problem by including a
%HMAC in the new \texttt{ADD\_ADDR} option. However, this HMAC uses the key
%that is exchanged in clear during the connection handshake.
With \tcpls, this
security concern can be addressed elegantly. First, Multipath \tcpls
would not need to exchange a key in clear. It uses cookies (random 128-bits
bitstrings) sent as Encrypted Extensions in the ServerHello during the
handshake, and utilized in \tcpls \texttt{JOIN} handshakes.
Second, in the case of Multipath TCP, the \texttt{ADD\_ADDR} and
\texttt{RM\_ADDR} option could be sent inside \tls records that are encrypted and
authenticated. The information would then reach MP\tcp using a new
\texttt{setsockopt}. Furthermore, since the \tls records are part of the bytestream,
they are reliably delivered in contrast with the new \texttt{ADD\_ADDR} option
that is transmitted unreliably, like all \tcp options, and thus needs to be
echoed.

\subsection{A More Secure \tcp Fast Open}

\tcp Fast Open (TFO) \cite{rfc7413,radhakrishnan2011tcp} is another recent
\tcp extension that allows sending data inside the SYN. TFO defines a
\tcp option that encodes a cookie. This cookie is used to prevent attacks
from spoofed IP addresses. When a client connects first to a server, it sends
an empty cookie but no data in the SYN packet. The server computes a cookie
bound (e.g. using a hash) to the client's IP address and returns it
in the SYN+ACK. For subsequent connections, the client sends its cookie in the SYN
and places data in the payload. The server validates the cookie and processes
the data since it comes from a legitimate IP address. However, the \tcp header
length limits the size of the TFO cookie. \tcpls could easily include
a longer cookie inside the \tls ClientHello within the SYN
payload. This solution would reduce the number of options in the \tcp header
and provide stronger protections against attackers. With this change, \tcpls
would support a 0-RTT connection establishment similar to QUIC. In datacenters and controlled environments, this would work well. However, measurements
are required before deploying that approach in enterprise and
wireless networks as some of them contain middleboxes that block \tcp Fast Open
\cite{paasch2016network}. This middlebox interference has also affected QUIC \cite{langley2017quic} and is thus not specific to \tcp.


\subsection{Pluginizing \tcpls}

PQUIC~\cite{de2019pluginizing} proposed an elegant approach to deploy
QUIC extensions. Instead of waiting for new client and server implementations,
PQUIC includes an eBPF virtual machine to implement new features as bytecode
that can be exchanged over the QUIC connection.
%\tcpls could also be ``pluginized'' like Pluginized QUIC~\cite{de2019pluginizing}. In a nutshell, Plugizined QUIC is a QUIC implementation that includes an eBPF virtual machine and is structured to accept extensions in eBPF bytecode.
%This bytecode that implemnents new protocol features can be loaded from
%disk or exchanged using one the streams of an existing QUIC connection. This
%enables a QUIC server to dynamically push protocol extensions to its clients.

A \tcpls implementation could also be pluginized. The Linux kernel
already includes an eBPF virtual machine~\cite{ebpf:2014}.
It has already been used to develop several types of
\tcp extensions \cite{brakmo2017tcp, tran2019beyond, tran2020beyond}
and recent versions of the mainline kernel allow loading congestion control schemes implemented in eBPF bytecode. \tcpls can transport eBPF bytecode using
\tls records as a second non-data stream.
An interesting research question would be to evaluate the
limits of such a dynamic extensibility? A first intuition to make \tcp's
extensibility mechanism independent of the \tcpls version would be to let \tcpls
communicating plugins to handle new \tcp options and control behaviours, such that the
supported \tcp extensibility capability is not frozen by a given \tcpls
version, but rather dependent on the set of plugins exchanged.

\subsection{Limits to Cross-Layer Integration}

QUIC and \tcpls show that there are benefits in integrating protocols at
adjacent layers. QUIC already integrates HTTP/3~\cite{draft-ietf-quic-http},
but will likely be used by other
applications~\cite{draft-ietf-dprive-dnsoquic,ssh-quic}
in the future. \tcp and \tls are already used by a wide range of
applications~\cite{anderson2019tls}. Should \tcpls also be tuned
for specific applications such as HTTP/3? What are the critical differences
between QUIC and \tcpls from a functionality viewpoint?

%Finally, we would expect to critisize QUIC in light of our \texttt{\tcpls}
%design. We will compare both QUIC and \tcpls, discuss the differences, and analyze
%and answer questions regarding the requirements in transport protocol for today's
%Internet applications.



%\subsection{Plugizing \tcpls}


%Tran shown that the Linux \tcp implementation
%could also be extended using eBPF~\cite{tran2019beyond}. In addition to the
%socketA recent paper


%This could be added to our prototype, but a more interesting
%research question would be to define a generic API that any \tcpls
%implementation would implement and expose to eBPF

\subsection{Middlebox Interference}

As explained earlier, the deployment of recent \tcp extensions has been
hindered by various types of
middleboxes~\cite{detal2013revealing,medina2004measuring,hesmans2013tcp,edeline2017first}.
QUIC already suffered from such problems~\cite{langley2017quic}
and many implementations fall back to \tls over \tcp when blocked by a firewall.
Researchers and application developers could include middlebox detection
techniques inside \tcpls. Consider a \tcpls client that copies its \syn header
within a \tcpls message alongside the early data, or makes it part of the
zero-rtt resumption ticket message. By comparing the received \tcp header with
the original one, the server would immediately and reliably detect the presence
of NAT, transparent proxies or other types of middleboxes.  It could then inform
the client and adjust the configuration of their stacks accordingly or fall back
to regular \tcp to preserve connectivity. If deployed, such a protocol would
enable researchers to understand the impact of middleboxes more accurately.

\subsection{Efficient \tcpls Implementations}

During the last years, researchers have proposed
to move transport protocol implementations to user-space
\cite{jeong2014mtcp,marinos2014network} to
kernel bypass techniques such as netmap \cite{rizzo2012revisiting} or
DPDK. In parallel, parts of \tls moved to the Linux kernel
\cite{borkmanncombining} for performance reasons.
With new results on SmartNICs \cite{firestone2018azure}, it would be
interesting to analyze the best architecture for new \tcpls implementations.
Given the enormous efforts on implementing QUIC\cite{quicimplem},
it would be exciting
to compare QUIC and \tcpls from a performance viewpoint. From a protocol
standpoint, performance advantages of combining those two layers may be
achieved from, for example, adjusting the size of \tls records based on the
current \tcp congestion
window to avoid fragmented records (non-fragmented records makes \tcpls'
design having a zero-copy code path). More generally, we expect many performance
benefits from a more intertwined \tls/\tcp transport protocol at the
cost of design complexity.



%\subsection{What are the Limits of Cross-Layers Integration ?}

%QUIC is a recent example of an integrated protocol since it combines
%transport functions with \tls and application layer functions (HTTP/3). This
%integration can improve performance, but what are its drawbacks ?


%\newpage

%We expect to investigate several research questions with \tcpls. First, we are
%interested in analyzing how far we can go into supporting \tcp's extensibility
%through our mechanism. Several of new \tcp features may require to inject eBPF
%bytecode to the kernel. It is still unclear how much of \tcp could be extended,
%and if \texttt{\tcpls} may play a role in incentivizing the linux kernel's maintainer
%to support more eBPF in \tcp's implementation, since we now offer a technique to
%propagate eBPF bytecode within authenticated and confidential session with a
%trusted server.

%Second, we expect to analyze several features of \tcpls, such as our
%connection migration, our multipath implementation and our failover mechanism.
%Answering the question about how much \tcpls can or and how fast it performs in
%case of a migration or a network outage.
