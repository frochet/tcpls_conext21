
% !TEX root = ./paper.tex
\tcp~\cite{rfc793} enables a client and a server to exchange data
over a connection that exposes a reliable bidirectional bytestream.
Dozens of \tcp extensions have been standardized and implemented~\cite{RFC7414}
while still preserving the same packet format. Most of these extensions rely on
\tcp options whose usage is negotiated during the three-way handshake.

\tcp was designed as an end-to-end protocol that is only used on endhosts.
Despite this architectural principle, network operators have deployed a variety
of middleboxes (e.g., firewalls, NATs, transparent proxies)~\cite{mCloud} that
sometimes interfere with \tcp or its extensions~\cite{medina2004measuring,
honda2011still, edeline2019bottom}. These in-path network functions make
assumptions about the content of \tcp packets, invalidating the initial
end-to-end paradigm of \tcp. Moreover, they do not always strictly follow the
\tcp specifications~\cite{honda2011still, hesmans2013tcp}, which may negatively
affect \tcp evolution and performances~\cite{edeline2020evaluating}.


In the 1990s and early 2000s, \tcp was mainly used directly by applications such
as HTTP, SMTP, FTP, telnet,~\ldots~However, when an application exchanges
plaintext using \tcp, it exposes its users to various privacy problems and
attacks. Initially, only banks and e-commerce websites considered the usage of
plaintext to be problematic and used cryptographic techniques to encrypt and
authenticate the information exchanged using Secure Socket Layer
(SSL)~\cite{draft-hickman-netscape-ssl}. At that time, using SSL was costly from
a performance viewpoint. During the last two decades, the situation completely
changed. Modern CPUs include specialized instructions that enable fast
encryption. Furthermore, the SSL standardization improved the protocol security.
The most recent version (\tls 1.3~\cite{rfc8446}) is considered to be more
secure than the previous ones, and initatives such as Let's Encrypt~\cite{aas2019let} have simplified the distribution of certificates.

\tls 1.3 brings several essential features compared to the previous versions. It
includes a secure handshake that allows negotiating the security parameters and
keys within one round-trip-time. Thanks to \tcp Fast Open
\cite{radhakrishnan2011tcp}, it is also possible to perform the secure handshake
during the \tcp handshake. Furthermore, it is also possible to exchange
application data during the handshake. The \tls 1.3 record layer protects all
application data with encryption and authentication. This record layer is
extensible, and the \tls record types are also encrypted to prevent
ossification.

During the last five years, the IETF has put a lot of effort
into designing and deploying a new transport protocol targeted at web
applications: \quic \cite{langley2017quic}. \quic provides a secure and
reliable delivery like the \tls/\tcp stack, but on top of \udp.
\quic
version 1.0 is being finalised~\cite{draft-ietf-quic-transport} and there are
more than a dozen implementations~\cite{quicimplem,marx2020same}. \quic is
already used in production as shown by recent measurement
studies~\cite{trevisan2020five}.

A growing number of applications are used on devices such as smartphones
attached to two or more networks. Users expect their applications to be
resilient to network failures.  With regular \tcp, applications need to
reestablish their connections when one of their network connection fails.
Multipath \tcp (\mptcp)~\cite{rfc8684,raiciu2012hard} is a \tcp extension
enabling a connection to use different paths. One of the main use cases for
\mptcp is to provide fast failovers on iPhones~\cite{bonaventure2016multipath}.
It also provides bandwidth aggregation by simultaneously using two or more
network paths to support one connection. Multipath extensions have also been
proposed for \quic~\cite{viernickel2018multipath,de2017multipath}. \quic version
1~\cite{draft-ietf-quic-transport} includes a connection migration capability
that supports failovers but not bandwidth aggregation.
