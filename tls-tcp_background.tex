
% !TEX root = ./paper.tex
The Transmission Control Protocol (\tcp)~\cite{rfc793} is the most
popular Internet transport protocol. It enables a client and a server
to exchange data over a connection that exposes a bidirectional
reliable bytestream. \tcp has evolved a lot since the first specification publication while still preserving the same packet format. Dozens of \tcp extensions have been standardised and implemented~\cite{RFC7414}. Several of these extensions rely on \tcp options and are negotiated during the three-way handshake.

The measurement studies carried out during the last decades, see, e.g.,~ \cite{paxson1994growth,claffy1993traffic,maier2009dominant,trevisan2020five}, have shown that \tcp was by far the most widely used protocols. It is
only recently that measurement studies have shown a growth in \udp
traffic~\cite{trevisan2020five}.

%\cite{RFC793}
%\cite{RFC7414} % roadmap extensions (dozen)

% tcp fast open article
%\cite{radhakrishnan2011tcp}
% mais problème (christoph)
%\cite{paasch2016network}

\tcp was designed as an end-to-end protocol that is only used on endhosts. Despite this architectural principale, network operators have deployed a variety of middleboxes (e.g., firewalls, NATs, transparent
proxies)~\cite{mCloud} that sometimes interfere with \tcp or its extensions~\cite{medina2004measuring, honda2011still, edeline2019bottom}.
These in-path network functions make assumptions about the content of \tcp
packets, invalidating the initial end-to-end paradigm of \tcp. Moreover, they do not always strictly follow the \tcp specifications~\cite{honda2011still, hesmans2013tcp}, which may negatively affect \tcp evolution and performances~\cite{edeline2020evaluating}.
% which

%\cite{honda2011still} % honda still possible to extend TCP

%- de très nombreux types de middleboxes
%\cite{medina2004measuring} % oldest middlebox paper   title={Measuring interactions between transport protocols and middleboxes},

%\cite{edeline2017first} % title={A first look at the prevalence and persistence of middleboxes in the wild},
%\cite{hesmans2013tcp} % {Are TCP extensions middlebox-proof?}},
%\cite{detal2013revealing} % tracebox, probablement pas utile


In the 1990s and early 2000s, \tcp was mainly used directly by applications such as HTTP, SMTP, FTP, telnet, ssh, \ldots However, when an application exchanges plaintext using \tcp, it exposes its users to various privacy problems and attacks. Initially, only banks and e-commerce websites considered the usage of plaintext to be problematic and used cryptographic techniques to encrypt and authenticate the information exchanged using Secure Socket Layer (SSL)~\cite{draft-hickman-netscape-ssl}. At that time, using SSL was costly from a performance viewpoint. During the last two decades, the situation completed changed. Modern CPUs include specialised instructions that enable fast encryption. Furthermore, the SSL standardisation improved the protocol security. The most recent version (\tls 1.3~\cite{rfc8446}) is considered to be much more secure than the previous ones. Furthermore, Let's Encrypt~\cite{aas2019let} has simplified the certificates distribution. Transport Layer Security (\tls) is on 90\% of the Alexa top 1M websites~\cite{holz2019era,holz2020tracking} and many other applications also
rely on \tls~\cite{anderson2019tls}. More importantly, modern applications
require both the reliability provided by \tcp and the security provided
by \tls.

\tls 1.3 brings several important features compared to the previous versions. It includes a secure handshake that allows to negotiate the security parameters and keys within one round-trip-time. Thanks to \tcp Fast Open \cite{radhakrishnan2011tcp}, it is also possible to perform the secure handshake during the \tcp handshake. Furthermore, it is possible to also exchange application data during the handshake. The \tls 1.3 record layer protects all application data with encryption and authentication. This record layer is extensible and to prevent ossification, \tls record types are also encrypted.

%TLS, 1.1 -> TLS.3

%\cite{rfc8446} TLS 1.3

%\cite{holz2019era} %   title={{The Era of TLS 1.3: Measuring Deployment and Use with Active and    Passive Methods}},
% plus récent
%\cite{holz2020tracking} % {Tracking the deployment of TLS 1.3 on the
                        % Web: A story of experimentation and
                        % centralization},

%TLS used for many applications
%\cite{anderson2019tls}
% let's ecnrypt justifie le succès
%\cite{aas2019let}

During the last five years, cloud vendors and the IETF have put a lot
of effort in designing and deploying a new transport protocol targeted
at web applications: \quic \cite{10.1145/3098822.3098842}. \quic
provides a secure and reliable delivery like the traditional \tls/\tcp
stack, but on top of \udp. This choice was motivated by two design
requirements. First, \quic must be able to go through middleboxes and most
of the deployed ones only support \tcp and \udp. Second, it must be
possible to implement \quic as a userspace library. \quic version 1.0 is being finalised~\cite{draft-ietf-quic-transport} and there are more than a dozen implementations~\cite{quicimplem,marx2020same}. \quic is used in
production as shown by recent measurement studies~\cite{trevisan2020five}.

% derniers drafts et papier google
%\cite{10.1145/3098822.3098842} % QUIC google
%\cite{draft-ietf-quic-transport}

% quicimplementations (bcp)
%\cite{quicimplem}

In addition to reliable and secure delivery, a growing number of applications
are used on devices such as smartphones that are attached to two or more
networks. Users expect their applications to be resilient to network failures.
With regular \tcp, applications need to reestablish their connections when one
of their network connection fails. Multipath \tcp (\mptcp)~\cite{rfc8684,raiciu2012hard} is a \tcp extension that enables a connection to use different paths. One of the main use cases for \mptcp is to provide fast failovers on iPhones~\cite{bonaventure2016multipath}. It also provides bandwidth aggregation by simultaneously using two or more network paths to support one connection. Multipath extensions have also been proposed
for \quic~\cite{viernickel2018multipath,de2017multipath}. \quic version
1.0~\cite{draft-ietf-quic-transport} includes a connection migration
capability that supports failovers but not bandwidth aggregation.


%Multipath transport (MPTCP et MPQUIC), deux use case failover et agregation

%\cite{rfc8684} % protocole

% threat mptcp
%\cite{rfc6181}

% mptpcp deployments
%\cite{bonaventure2016multipath}

% MPTCP implem
%\cite{raiciu2012hard}

%mptcp path management
%\cite{hesmans2015smapp}

% mptcp API
%\cite{hesmans2016enhanced}

% mptcp scheduler

%MPQUIC:


%Protocol extensibility ? Pas encore sur que c'est le meilleur endroit
%pour expliquer cela.

%eBPF

%d'abord \cite{brakmo2017tcp} et a été bcp amélioré depuis avec
%maintenant contrôle de congestion

%\cite{tran2019beyond} ou \cite{tran2020beyond}

%\cite{de2019pluginizing} % PQUIC

% mentionne d'autres efforts liés ? ou alors on le met comme surprise
% plus loin
%\cite{narayan2018restructuring}
%  title={Restructuring endpoint congestion control},

%Here comes background on TCP/TLS/UDP/QUIC/MIDDLEBOXES/BPF
