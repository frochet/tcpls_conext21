
% !TEX root = ./paper.tex
%Notion de streams
%Expliquer comment TLS et QUIC sont intégrés.

\tcp was designed as an end-to-end protocol that is only used on end-hosts.
\tcp includes flow-control, supports various retransmission techniques to cope
with packet losses, and congestion control mechanisms. \tcp stacks have evolved
a lot during the last decades. Most extensions to \tcp leverage the \tcp
\texttt{Options} space, which is limited to 40 bytes. Unfortunately, the \tcp
designers did not foresee that many \tcp extensions would be standardized.
Today, the \tcp header size is a constraint. The IETF has discussed this
problem for several years, but the latest attempt to solve
it~\cite{draft-ietf-tcpm-tcp-edo-10} has not yet been implemented by major
\tcp stacks.

Modern applications rarely use \tcp alone. They often combine \tcp with Transport Layer Security (\tls). \tls 1.3~\cite{rfc8446} brings several essential features compared to the previous versions. It includes a secure handshake that allows negotiating the security parameters and keys within one round-trip time. Thanks to \tcp Fast Open~\cite{radhakrishnan2011tcp}, it is also possible to perform the secure handshake during the \tcp handshake. Furthermore, it is also possible to exchange application data during the handshake. The \tls 1.3 record layer protects all application data with encryption and authentication. This record layer is extensible, and the \tls record types are also encrypted to prevent ossification.

Despite this architectural principle, network operators have deployed a variety
of middleboxes (e.g., firewalls, NATs, transparent proxies)~\cite{mCloud} that
sometimes interfere with \tcp or its extensions~\cite{medina2004measuring,
honda2011still, edeline2019bottom}. These in-path network functions make
assumptions about the \tcp packets content, invalidating the initial \tcp
end-to-end paradigm. Moreover, they do not always strictly follow the
\tcp specifications~\cite{honda2011still, hesmans2013tcp}, which may negatively
affect \tcp's evolution and performance~\cite{edeline2020evaluating}. These
middleboxes severely limit the \tcp extensibility. Multipath \tcp~\cite{rfc8684,raiciu2012hard} (\mptcp) managed to cope with this interference, but at the price of increased complexity with notably the utilization of a second level of checksum to detect middlebox interference~\cite{raiciu2012hard,hesmans2013tcp}.

\begin{table}[!t]
  \setlength\tabcolsep{3pt}
  \small
  \begin{tabular}{lccccc}
    \toprule
    & \tcp & \mptcp & \tls/\tcp & \quic & \tcpls \\
    \midrule
    Reliability \& cong. control & \checkmark & \checkmark & \checkmark &
    \checkmark & \checkmark \\
    Message conf. and auth.&  \xmark & \xmark & \checkmark & \checkmark &
    \checkmark \\
    Failover &  \xmark & \checkmark &\xmark & (\checkmark) & \checkmark \\
    HoL blocking avoidance & \xmark & \xmark & \xmark & \checkmark &
    (\checkmark) \\
    Streams & \xmark &  \xmark & \xmark & \checkmark & \checkmark \\
    Connection migration & \xmark & (\checkmark) & \xmark & (\checkmark) &
    \checkmark \\
    Concurrent paths & \xmark & \checkmark & \xmark & \xmark & \checkmark \\
    \bottomrule
  \end{tabular}
  \caption{Comparison of \xmark\ the services not offered, (\checkmark)\
  partially available, and \checkmark\ offered by protocols.}
  \label{table:tcpquictcpls}
\end{table}

To counter middlebox interference, Google took a different approach by 
developing an entirely new secure transport protocol, 
\quic~\cite{langley2017quic}, combining \tcp, \tls, and HTTP features in a 
single protocol implemented in a userspace library and running above \udp. 
\quic prevents middlebox interference by encrypting and authenticating the 
data, but also most of the control information such as the acknowledgments 
except a very small header. \quic leverages \tls 1.3~\cite{rfc8446} to 
negotiate the security keys and authenticate the server. \quic has a flexible 
framing system and packets are encrypted separately, in a similar manner to 
\tls records. \quic allows the application to use different datastreams over a 
single connection such \sctp~\cite{rfc4960} or Structured 
Streams~\cite{ford2007structured}. \quic has been adopted by the industry and 
standardized within the IETF~\cite{rfc9000}. There are more than a dozen \quic 
implementations~\cite{quicimplem,marx2020same}. \quic is already used in 
production as shown by recent measurement studies~\cite{trevisan2020five}.

Table~\ref{table:tcpquictcpls} summarises the main features of \tcp,\mptcp, 
\tls over \tcp, and \quic. All protocols include reliability and congestion 
control. \tls and \quic provide the same security features. \mptcp efficiently 
supports failovers and \quic includes a connection migration capability. \quic 
supports stream and prevents Head-of-Line (HoL) blocking. \mptcp is the only
standardised protocol that is able to aggregate different paths bandwidth. 
\mptcp implementations include several path management 
strategies~\cite{hesmans2015smapp,hesmans2016enhanced} to control the different 
paths utilization. \quic allows the client to migrate its connection, although 
it is rarely exposed to applications in current implementations. Multipath QUIC 
extensions enabling the use of concurrent paths in QUIC have recently been 
proposed. \todo{ref}
%(mp): Actually, none really offer that to my knowledge. I guess that's why
%Florentin
%added the (v).
%enable the applications to decide when connections should be migrated.





%A growing number of applications are used on devices such as smartphones
%attached to two or more networks. Users expect their applications to be
%resilient to network failures.  With regular \tcp, applications need to
%reestablish their connections when one of their network connection fails.
%Multipath \tcp (\mptcp)~\cite{rfc8684,raiciu2012hard} is a \tcp extension
%enabling a connection to use different paths. One of the main use cases for
%\mptcp is to provide fast failovers on iPhones~\cite{bonaventure2016multipath}.
%It also provides bandwidth aggregation by simultaneously using two or more
%network paths to support one connection. Multipath extensions have also been
%proposed for \quic~\cite{viernickel2018multipath,de2017multipath}. \quic version
%1~\cite{draft-ietf-quic-transport} includes a connection migration capability
%that supports failovers but not bandwidth aggregation.

%\todo{cite secure stream, tcpcrypt}

%\todo{Explain TLS1.3 and what it brings}
%\todo{Explain QUIC and how it uses TLS1.3}
%\todo{discuss middleboxes and mention that QUIC is blocked in some parts of the Internet - do we have numbers ?}

%\tcp~\cite{rfc793} enables a client and a server to exchange data
%over a connection that exposes a reliable bidirectional bytes stream.
%Dozens of \tcp extensions have been standardized and implemented~\cite{RFC7414}
%while still preserving the same packet format. Most of these extensions rely on
%\tcp options whose usage is negotiated during the three-way handshake.




%In the 1990s and early 2000s, \tcp was mainly used directly by applications such
%as HTTP, SMTP, FTP, telnet,~\ldots~However, when an application exchanges
%plaintext using \tcp, it exposes its users to various privacy problems and
%attacks. Initially, only banks and e-commerce websites considered the usage of
%plaintext to be problematic and used cryptographic techniques to encrypt and
%authenticate the information exchanged using Secure Socket Layer
%(SSL)~\cite{draft-hickman-netscape-ssl}. At that time, using SSL was costly from
%a performance viewpoint. During the last two decades, the situation completely
%changed. Modern CPUs include specialized instructions that enable fast
%encryption. Furthermore, the SSL standardization improved the protocol security.
%The most recent version (\tls 1.3~\cite{rfc8446}) is considered to be more
%secure than the previous ones, and initiatives such as Let's Encrypt~\cite{aas2019let} have simplified the distribution of certificates.


%into designing and deploying a new transport protocol targeted at web
%applications: \quic \cite{langley2017quic}. \quic provides a secure and
%reliable delivery like the \tls/\tcp stack, but on top of \udp.
%\quic
